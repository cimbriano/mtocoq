\documentclass[10pt,  onecolumn]{article}
\usepackage{enumerate}
\usepackage{url}

%To use \citeauthor
\usepackage[numbers]{natbib}

\begin{document}

\title{Coq Formalization of Memory Trace Oblivious Execution}
\author{ Chris Imbriano, Andrew Lohr \\
\{imbriano,  alohr1\}@umd.edu }
\date{May 2013}

\maketitle

% General Notes
%Your write up should be 5-15 pages, as necessary.


\begin{abstract}
% Summarizing your motivation and your accomplishments.
In a traditional security model, known as non-interference, a program's inputs and outputs are given security labels indicating that data's sensitivity; \emph{low} and \emph{high}.
\emph{Low} variables are those data that are public or can be infrared from publicly available information, whereas \emph{high} variables are secret or private data.
The non-interference model seeks to maintain the secrecy of \emph{high} variables by ensuring that all program executions with identical \emph{low} (public) inputs produce the same \emph{low} outputs.

While non-interference is a strong security model in theory, in practice real executing programs "leak" information about its data through a number of channels - timing, electromagnetic radiation, power consumption, and execution trace, to name a few.
\citeauthor{mtope} presented a type system to efficiently address information leakage via a program execution trace.
The proofs presented by \citeauthor{mtope} are presented in human-readable format but lack a corresponding proof certificate from verified proof assistant.

We present the foundation necessary to implement and verify the proofs by \citeauthor{mtope}.
In section \emph{SECTION LABEL}, we outline our approach.
In section \emph{SECTION LABEL}, we discuss challenges both technical and not.
\emph{FLESH THIS PART OUT}

\end{abstract}

\section{Introduction}
\label{sec:introduction}
% Summarizing your motivation and your accomplishments.
\cite{techreport}


% The intermediate sections should contain full details about what you did.
\section {Approach}

\section{Challenges}

% Proofs were written by humans for humans.
% Many times there we're "abuses of notation", "lemma 5 is proved using a similar technique as lemma 4"
	% which resulted in requiring more infrastructure/definitions than we initially envisioned
	

%- Duplication in definitions, lemmas, proofs for traces and trace patterns
%
%- Choosing to represent trace equivalence and trace pattern equivalence as Relations instead of functions (because they're only used in a relational way)
%
%- Handling of arrays
%	* Will this constrain extending the type system to allow each element of an array to occupy its own OramBank?
%
%- Many variables to keep track of.
%	* A lot of quantified variables in the statement of the theorem

\section{Related Work}
\label{sec:relatedwork}
% Research papers should also include a related work section. 

\section{Conclusion}
\label{sec:conclusion}
% End with a conclusion putting the project in perspective and mentioning open problems


%Issues/Difficulties/Notes
%
%- Duplication in definitions, lemmas, proofs for traces and trace patterns
%
%- Choosing to represent trace equivalence and trace pattern equivalence as Relations instead of functions (because they're only used in a relational way)

\bibliographystyle{plainnat}
\bibliography{report}


%\begin{thebibliography}{9}
%% bibliography of cited papers.
%	\bibitem{itemname} aaaaaaaaaaaa
%\end{thebibliography}

\end{document}