\documentclass[10pt,  onecolumn]{article}
\usepackage{enumerate}
\usepackage{url}

% Fixes issues with older encoding type that doesn't display < and > correctly.
\usepackage[T1]{fontenc}

% To put a border around figures
\usepackage{ mdframed }

% To use inline teletype command \texttt{}
\usepackage{courier}

%To use \citeauthor
\usepackage[numbers]{natbib}

% For using the listings package (to write code)
\usepackage{listings}
\usepackage{color}

\lstset{ %
	backgroundcolor=\color{white},   % choose the background color; you must add \usepackage{color} or \usepackage{xcolor}
	basicstyle=\footnotesize,        % the size of the fonts that are used for the code
	breakatwhitespace=false,         % sets if automatic breaks should only happen at whitespace
	breaklines=true,                 % sets automatic line breaking
	captionpos=b,                    % sets the caption-position to bottom
	deletekeywords={...},            % if you want to delete keywords from the given language
	escapeinside={\%*}{*)},          % if you want to add LaTeX within your code
	extendedchars=true,              % lets you use non-ASCII characters; for 8-bits encodings only, does not work with UTF-8
	frame=single,                    % adds a frame around the code
	keepspaces=true,                 % keeps spaces in text, useful for keeping indentation of code (possibly needs columns=flexible)
	keywordstyle=\color{blue},       % keyword style
	language=ML,                 % the language of the code
	morekeywords={*,...},            % if you want to add more keywords to the set
	numbers=left,                    % where to put the line-numbers; possible values are (none, left, right)
	numbersep=5pt,                   % how far the line-numbers are from the code
	rulecolor=\color{black},         % if not set, the frame-color may be changed on line-breaks within not-black text (e.g. comments (green here))
	showspaces=false,                % show spaces everywhere adding particular underscores; it overrides 'showstringspaces'
	showstringspaces=false,          % underline spaces within strings only
	showtabs=false,                  % show tabs within strings adding particular underscores
	stepnumber=1,                    % the step between two line-numbers. If it's 1, each line will be numbered
	tabsize=2,                       % sets default tabsize to 2 spaces
	title=\lstname                   % show the filename of files included with \lstinputlisting; also try caption instead of title
}



\begin{document}

\title{Coq Formalization of Memory Trace Oblivious Execution}
\author{ Chris Imbriano, Andrew Lohr \\
\{imbriano,  alohr1\}@umd.edu }
\date{May 2013}

\maketitle

% General Notes
%Your write up should be 5-15 pages, as necessary.


\begin{abstract}
% Summarizing your motivation and your accomplishments.
In a traditional security model, known as non-interference, a program's inputs and outputs are given security labels indicating that data's sensitivity; \emph{low} and \emph{high}.
\emph{Low} variables are those data that are public or can be infrared from publicly available information, whereas \emph{high} variables are secret or private data.
The non-interference model seeks to maintain the secrecy of \emph{high} variables by ensuring that all program executions with identical \emph{low} (public) inputs produce the same \emph{low} outputs.

While non-interference is a strong security model in theory, in practice real executing programs "leak" information about its data through a number of channels - timing, electromagnetic radiation, power consumption, and execution trace, to name a few.
\citeauthor{mtope} presented a type system to efficiently address information leakage via a program execution trace.
The proofs presented by \citeauthor{mtope} are presented in human-readable format but lack a corresponding proof certificate from verified proof assistant.

We present the foundation necessary to implement and verify the proofs by \citeauthor{mtope}.
In section \emph{SECTION LABEL}, we outline our approach.
In section \emph{SECTION LABEL}, we discuss challenges both technical and not.
\emph{FLESH THIS PART OUT}

\end{abstract}

\section{Introduction}
\label{sec:introduction}
% Summarizing your motivation and your accomplishments.


\section {Overview}
% The intermediate sections should contain full details about what you did.

We will begin by giving an overview of the Coq project files and the contents within each.
It will be assumed that the reader is familiar with the definitions and concepts presented in \emph{Memory Trace Oblivious Program Execution} (MTOPE).
Then, we will describe in more detail each file's contents with emphasis on definitions which differ substantially from those provided in the paper.


\section{ Project Structure }

% Include a section? Sflib, strong induction technique?
%	\subsection{ External Requirements }


% TODO Maybe these shouldn't be subsections, but some form of list instead?
	\subsection{ Core Definitions }
	The foundation of our definitions are found in a file called core.v.
	Here is where we define all structures common to that will be used in downstream definitions.
	For the most part the definitions here follow one-to-one from those in the paper, however the definition of statement required special handling.

	Below we reproduce the relevant definitions from the paper and our Coq definitions.


\begin{figure}
\caption{ MTPOE definitions for Statements and Programs }
\label{fig:mto_rule_while}

	\begin{mdframed}
	Numbers $n \in \textbf{Nat}$\\
	ORAM bank $o \in \textbf{ORAMbank}$\\
	Locations $p ::= n | o$\\
	Statements $ s ::= skip | x := e | x[e] := e | if(e, S, S) | while(e, S)$\\
	Programs $ S ::= p:s | S;S$
	\end{mdframed}

\end{figure}

\begin{figure}
\caption{ Typing Judgement for statements : while}
\label{fig:while_judgement}
\[
\frac{ \Gamma \vdash e : Nat \; l; T_{1}   \;\;\;\;  \Gamma, l_{0} \vdash S;T_{2}  \;\;\;\;   l \sqcup l_{0} \sqsubseteq L }
	{ \Gamma, l_{0} \vdash p:\textbf{while}(e,S);\textbf{Loop}(p,T_{1},T_{2}) }
\]
\end{figure}

\begin{figure}
\caption{Coq definitions for statements, programs and labeled statements}
\label{fig:coq_statements}
\begin{lstlisting}
Inductive statement : Type :=
  | skip : statement
  | assign: variable -> expression -> statement
  | arrassign: variable -> expression -> expression -> statement
  | stif: expression -> program -> program -> statement
  | stwhile: expression -> program -> statement

with labeledstatement : Type :=
  | labline : location -> statement -> labeledstatement

with program : Type :=
  | oneLineProg : labeledstatement -> program
  | progcat : program -> program -> program.
\end{lstlisting}
\end{figure}



Figure \ref{fig:while_judgement} shows the typing rule for a \texttt{while} statement.
Notice that even though figure \ref{fig:mto_rule_while} shows the definition of a while statement to be without a location, ie. \texttt{while(e,S)}, the typing judgement for statements types a while statement augmented with a location \texttt{p:while(e,S)}, which is structurally a "Program".
The typing rule requires this information since the type of trace pattern produced depends on a while statements label.
For this reason, we define \texttt{statement} in Coq to be analogous to the paper's definition, but we introduce a labeled statement.
Doing this further requires writing the typing rules in terms of labeled statements, rather than plain statements as is done by the paper.

Another deviation from the paper's definitions is in the implementation of the operator $\sqsubseteq$.
In all but one case, this operator compares two security labels, each of which can be either $L$ or an Oram Bank number - definition shown in figure \ref{fig:mto_labeldef}.
However, in the typing rule for, what the paper refers to as, a labeled statement one of the premises abuses the $\sqsubseteq$ operator by comparing a label, $l_{0}$ to a location. The rule is shown in figure \ref{fig:mto_rule_statement}

\begin{figure}
\caption{Formal definition for Labels}
\label{fig:mto_labeldef}
	\begin{mdframed}
	Labels $l \in \textbf{SecLabels} = \{L\} \cup \textbf{ORAMbanks}$
	\end{mdframed}
\end{figure}

\begin{figure}
\caption{ Typing rule for labeled statements under the judgment for Programs}
\label{fig:mto_rule_statement}
\[
\frac{ \Gamma, l_{0} \vdash s;T \;\;\; l_{0} \sqsubseteq p }
	{ \Gamma, l_{0} \vdash p:s;(\textbf{Fetch} p)@T}
\]
\end{figure}

\begin{figure}
\caption{ Coq definition of the $\sqsubseteq$ operator }
\label{fig:coq_label_le}
\begin{lstlisting}
Inductive label_le : label -> label -> Prop :=
  | fst_low : forall   l, label_le low l
  | snd_high: forall l n, label_le l   (o_high n).

Inductive label_le_rhslocation : label -> location -> Prop :=
  | highloc : forall l n, label_le_rhslocation l   (oram n)
  | lowlab  : forall   p, label_le_rhslocation low p.
\end{lstlisting}

\end{figure}

Figure \ref{fig:coq_label_le} shows our proposed implementation of this operator.
We essentially have two versions of the $\sqsubseteq$ relation, one which is the expected operation of comparing two security labels, and the other of which compares a label to a location.




\subsection{ Typing and Semantics }

The definitions and structures required for typing and semantics are located in Coq file named \texttt{typing.v} and \texttt{semantics.v}.
We'll omit discussion of the mostly straightforward parts of these two sections and focus on clarifying naming conventions and describing implementation difficulties or judgement calls.

% Naming convention - labeled types and labeled values
The basic types for values in this type system are $Nat$ and $Array$.
However, since the goal of this work is to maintain the security of sensitive data, the values types are augmented with security labels as discussed in the previous section.
The paper uses the term Types to refer to one of $Nat$ or $Array$ augmented with a security label, whereas our implementation uses the terms \texttt{labeledType} and \texttt{labeledValue} as shown in figure \ref{fig:coq_labeled_things}.

\begin{figure}
\caption{Plain basic type sand augmented with security labels}
\label{fig:coq_labeled_things}

\begin{lstlisting}
Definition mtonat   : Type := nat.
Definition mtoarray : Type := mtonat -> option mtonat.

Inductive labeledValue: Type :=
  | vint : mtonat   -> label -> labeledValue
  | varr : mtoarray -> label -> labeledValue.

Inductive labeledType : Type :=
  | larr : label -> labeledType
  | lnat : label -> labeledType.
\end{lstlisting}
\end{figure}



% Memory and Gamma are total functions
Both the typing definitions and semantics definitions require a mapping from variables to some value.
In typing this map is called $\Gamma$, or the typing environment,  which maps variables to types.
In semantics this is a representation of memory, $M$, which maps variables to values which can have type $Array \; l$ or $Nat \; l$, where $l$ is a security label.
In the paper, these maps are treated as partial-functions that are defined for some finite domain and undefined all inputs outside that domain.
In Coq however, we're forced to define total functions and we do so by returning an option type as shown in figure \ref{fig:coq_total_functions}.


\begin{figure}
\caption{ Coq definitions for environment and memory }
\label{fig:coq_total_functions}
\begin{lstlisting}
Definition environment : Type := variable -> (option labeledType).
Definition      memory : Type := variable -> (option labeledValue).
\end{lstlisting}
\end{figure}



% Judgement call - Mutable arrays as functions that get updated
The semantic rules for array assignment necessitate the modeling of mutable arrays.
Since Coq is a functional language, our arrays are implemented as functions which take an index and return a value - either the value at the index if it is in bounds, or zero if the index is out of bounds.
In order to support mutability, updates to an array are implemented as a wrapping function around an existing array.
When the array, or equivalently the wrapping function, is applied to an index, if that index matches that given to the wrapping function as part of the update, then the application is "caught" at the wrapping function and a value is returned immediately.
Otherwise, if the index is not that related to the last update, the wrapping function passes along the index to the last wrapped array.
Figure \ref{fig:coq_arrays} shows this implementation.
It should be noted, that part of the assumptions of the paper is that all variables exist in memory and in the typing environment 

\begin{figure}
\caption{ Coq implementation of array get and update}
\label{fig:coq_arrays}

\begin{lstlisting}
Definition mtoarrget (m:mtoarray) (n:mtonat) : mtonat :=
  match m n with
  | None => O
  | Some a => a
  end.

Definition mtoarrupd (m:mtoarray) (n1 n2:mtonat) : mtoarray :=
  match m n1 with
  | None => m
  | Some a =>
      fun ind =>
        if   (beq_nat ind n1)
        then (Some n2)
        else (m ind)
  end.
\end{lstlisting}

\end{figure}

% Choosing to represent trace equivalence and trace pattern equivalence as Relations instead of functions (because they're only used in a relational way)

One aspect of implementation in which we initially went in the wrong direction was in our approach to Trace Pattern equivalence.
Our initial idea was to try to write a function which given two Trace Patterns would return a boolean if the two were equivalent according to the rules shown in figure \ref{fig:mto_TracePat_equiv}.
However, since Trace Pattern equivalence is a proposition, (Prop type in Coq), we were attempting to produce a function from Prop to boolean, which 


\begin{figure}
\caption{Trace Pattern Equivalence rules}
\label{fig:mto_TracePat_equiv}
\[
\frac {T ~_L T}
        {\epsilon @T  \sim_{L}T@\epsilon  \sim_{L} T}
\]
\[
\epsilon \sim_{L}  \epsilon
\]
\[
\frac {T_1 \sim_L T_2 \;\;\; T_2 \sim_L T_3}
        {T_1 \sim_L T_3}
\]
\[
\frac {T_1 \sim_L T'_1 \;\;\; T_2 \sim_L T'_2}
        {T_1@T_2 \sim_L T'_1@T'_2}
\]
\[
(T_1@T_2)@T_3 \sim_L T_1@(T_2@T_3)
\]
\[
o \sim_L o
\]
\[
\textbf{Fetch}\;p \sim_L \textbf{Fetch}\;p
\]
\[
\textbf{Read}\;x \sim_L \textbf{Read}\;x
\]


\end{figure}










\subsection{ Lemmas and Proofs }

When proving theorem one, we found a stond induction prinicple that we included. \cite{strongind}
This was used twice in our proof. First, when considering the case in which the program is a single if statement, we performed strong induction on the number of statements in the program. In the case of a single while loop, the technical report makes use of induction over the number of times a while loop is executed. This was difficult to formalize in Coq, so we decided to do strong induction on the length of the trace from executing the while loop.



% Proofs were written by humans for humans.
% Many times there we're "abuses of notation", "lemma 5 is proved using a similar technique as lemma 4"
	% which resulted in requiring more infrastructure/definitions than we initially envisioned


%- Duplication in definitions, lemmas, proofs for traces and trace patterns
%
%- Handling of arrays
%	* Will this constrain extending the type system to allow each element of an array to occupy its own OramBank?
%
%- Many variables to keep track of.
%	* A lot of quantified variables in the statement of the theorem

\section{Related Work}
\label{sec:relatedwork}
% Research papers should also include a related work section.

\section{Conclusion}
\label{sec:conclusion}
% End with a conclusion putting the project in perspective and mentioning open problems


%Issues/Difficulties/Notes
%
%- Duplication in definitions, lemmas, proofs for traces and trace patterns
%
%- Choosing to represent trace equivalence and trace pattern equivalence as Relations instead of functions (because they're only used in a relational way)
\cite{sf}
\cite{mtope}
\cite{techreport}
\cite{strongind}


\bibliographystyle{plainnat}
\bibliography{report}


%\begin{thebibliography}{9}
%% bibliography of cited papers.
%	\bibitem{itemname} aaaaaaaaaaaa
%\end{thebibliography}

\end{document}